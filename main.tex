\documentclass{article}

\usepackage[final]{nips_2018}

\usepackage[utf8]{inputenc} % allow utf-8 input
\usepackage[T1]{fontenc}    % use 8-bit T1 fonts
\usepackage{hyperref}       % hyperlinks
\usepackage{url}            % simple URL typesetting
\usepackage{booktabs}       % professional-quality tables
\usepackage{amsfonts}       % blackboard math symbols
\usepackage{nicefrac}       % compact symbols for 1/2, etc.
\usepackage{microtype}      % microtypography
\usepackage{amsmath}
\usepackage{mathtools}

\usepackage{subfiles}

\usepackage{graphicx}

\title{CECS 551 Project}

\author{%
  Jared R.~Coleman\\
  Computer Engineering and Computer Science\\
  California State University, Long Beach\\
  Long Beach, CA 90802 \\
  \texttt{jared.coleman@student.csulb.edu} \\
  \And 
  Taina G.D.~Coleman\\
  Computer Engineering and Computer Science\\
  California State University, Long Beach\\
  Long Beach, CA 90802 \\
  \texttt{taina.coleman@student.csulb.edu} \\
  \And 
  Ian M. ~Schenck\\
  Computer Engineering and Computer Science\\
  California State University, Long Beach\\
  Long Beach, CA 90802 \\
  \texttt{ian.schenck@student.csulb.edu} \\
}

\begin{document}

\maketitle

\begin{abstract}
   Since its introduction in 2014, the Generative Adverserial Network (GAN) has been adapted and improved upon at Machine Learning conferences all over the world. In this short survey, we discuss the fundamentals of GAN and one of its successors, the Wasserstein GAN (WGAN).
\end{abstract}

\subfile{introduction}
\subfile{gan}
\subfile{wgan}
\subfile{gcn}

\section{Conclusion \& Future Work}
  \subsection{Generative Adversarial Networks}
  \subsection{Graph Convolutional Networks}

\bibliographystyle{plain}
\bibliography{references}

\end{document}

