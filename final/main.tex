\documentclass{article}

\usepackage[final]{nips_2018}

\usepackage[utf8]{inputenc} % allow utf-8 input
\usepackage[T1]{fontenc}    % use 8-bit T1 fonts
\usepackage{hyperref}       % hyperlinks
\usepackage{url}            % simple URL typesetting
\usepackage{booktabs}       % professional-quality tables
\usepackage{amsfonts}       % blackboard math symbols
\usepackage{nicefrac}       % compact symbols for 1/2, etc.
\usepackage{microtype}      % microtypography
\usepackage{amsmath}
\usepackage{mathtools}

\usepackage{subfiles}

\usepackage{graphicx}
\usepackage{subcaption}

\title{CECS 551 Final Report}
\author{%
  Jared R.~Coleman\\
  Computer Engineering and Computer Science\\
  California State University, Long Beach\\
  Long Beach, CA 90802 \\
  \texttt{jared.coleman@student.csulb.edu} \\
  \And 
  Taina G.D.~Coleman\\
  Computer Engineering and Computer Science\\
  California State University, Long Beach\\
  Long Beach, CA 90802 \\
  \texttt{taina.coleman@student.csulb.edu} \\
  \And 
  Ian M. ~Schenck\\
  Computer Engineering and Computer Science\\
  California State University, Long Beach\\
  Long Beach, CA 90802 \\
  \texttt{ian.schenck@student.csulb.edu} \\
}

\begin{document}
\stepcounter{equation}
\setcounter{equation}{0}

\maketitle

\begin{abstract}
  Neural networks, particularly relating to image and text classification, have boomed in the 21\textsuperscript{st} century. State-of-the-Art models now achieve better-than-human performace at some image classification tasks. It wasn't until very recently, however, that neural networks began to show promise for \textit{graph datasets}. Researches are interested in applying machine learning techniques to graph datasets because they recognize that the relationship between objects can potentially carry alot of information about the object itself. For example, in a Social Network, it's probable that people have alot in common with their closest connections. Traditional Neural Networks aren't equipped to deal with the kind of relationship data available in graphs. This paper is an explanation and attempted replication of the 2017 paper by Kipf and Welling~\cite{Kipf2016} that introduces Graph Convolutional Networks - a neural network model for semi-supervised learning on graphs.
\end{abstract}

\subfile{introduction} 
\subfile{gcn}
\subfile{experiments}
\subfile{conclusion}

  
\bibliographystyle{plain}
\bibliography{references}

\end{document}

