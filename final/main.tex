\documentclass{article}

\usepackage[final]{nips_2018}

\usepackage[utf8]{inputenc} % allow utf-8 input
\usepackage[T1]{fontenc}    % use 8-bit T1 fonts
\usepackage{hyperref}       % hyperlinks
\usepackage{url}            % simple URL typesetting
\usepackage{booktabs}       % professional-quality tables
\usepackage{amsfonts}       % blackboard math symbols
\usepackage{nicefrac}       % compact symbols for 1/2, etc.
\usepackage{microtype}      % microtypography
\usepackage{amsmath}
\usepackage{mathtools}

\usepackage{subfiles}

\usepackage{graphicx}
\usepackage{subcaption}

\title{CECS 551 Midterm Report}
\author{%
  Jared R.~Coleman\\
  Computer Engineering and Computer Science\\
  California State University, Long Beach\\
  Long Beach, CA 90802 \\
  \texttt{jared.coleman@student.csulb.edu} \\
  \And 
  Taina G.D.~Coleman\\
  Computer Engineering and Computer Science\\
  California State University, Long Beach\\
  Long Beach, CA 90802 \\
  \texttt{taina.coleman@student.csulb.edu} \\
  \And 
  Ian M. ~Schenck\\
  Computer Engineering and Computer Science\\
  California State University, Long Beach\\
  Long Beach, CA 90802 \\
  \texttt{ian.schenck@student.csulb.edu} \\
}

\begin{document}
\stepcounter{equation}
\setcounter{equation}{0}

\maketitle

\begin{abstract}
  Deep learning, particularly relating to image and text classification, has boomed in the 21\textsuperscript{st} century. State-of-the-Art models now achieve better-than-human performace at image classification tasks. Deep learning on graphs, however, has not been possible until recently. There is good reason to use the deep learning method on graph datasets. Social Networks, for example, have billions of nodes (people) and even more edges (relationships between people). It's obvious that a person's relationships can potentially carry alot of information about them (who they are, what they like, etc.). Traditional Deep-Learning models aren't equipped to deal with the kind of relationship data available in graphs. Graph Convoultional Networks were introduced by \cite{Kipf2016} in 2017. This paper is an explanation and attempted replication of their work.
\end{abstract}

\subfile{introduction} 
\subfile{gcn}
\subfile{experiments}
\subfile{conclusion}

  
\bibliographystyle{plain}
\bibliography{references}

\end{document}

