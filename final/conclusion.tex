\section{Conclusion}

Graph Convolutional Networks, as proposed in~\cite{Kipf2016}, provide an efficient and effective technique for semi-supervised learning on graph datasets. In this paper, decent results were shown on three different citation networks, although there were clear problems with overfitting. Improved regularization techniques sepcifically designed for graph neural networks are a possible area of future research. Also, GCNs are very shallow compared to the most successful traditional neural networks. This is a limitation of this model, since adding more layers is equivalent to increasing the range that nodes spread their features. Graph Convolutional Networks are just the beginning of a promising new area of research. 